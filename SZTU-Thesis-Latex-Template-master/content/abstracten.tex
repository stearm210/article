%!TEX root = ../sztuthesis_main.tex

\addcontentsline{toc}{section}{Abstract}

\keywordsen{Crowd counting; CNN framework; ATRM relation module; top-k relationships}
\categoryen{TP391}
\begin{abstracten}
    Crowd counting or crowd density estimation Estimation is a technique for computing or estimating the number of people in an image via a computer and can be used in security or other scenarios. In this paper, we perform crowd counting using top-k relational network, a CNN framework based on mixed truth values. At the same time, feature representation is enhanced by using top-k dependency between pixels via adaptive filtering mechanism using the adaptive top-k relation module atrm. The top-k relationship of each pixel position is obtained by calculating the similarity between two pixels. Next, a weight normalization method of the adaptive filtering mechanism is used to adaptively eliminate the influence of low-correlation positions in the top-k relation. Finally, the weight attention mechanism is used to ensure that the atrm pays attention to the position with the higher weight in relation to the top-k.
\end{abstracten}